\documentclass[
	bibliography=totoc,
	listof=totoc,							% falls Index verwendet, ergänze "index=totoc" zu den Optionen
	BCOR=5mm,							% Rand für Bindung: 5mm
	DIV=12]{scrbook}
\usepackage{bibgerm}       					% deutsche Literaturverzeichnisse
\usepackage[utf8]{inputenc}		       			% Umlaute im Text
\usepackage{graphicx} 						% Einfügen von Grafiken  - für PDF-Latex: .pdf und .png (.jpg möglich, sollte aber vermieden werden)
\usepackage{setspace}
\usepackage{fancyvrb, listings}          			% Für Code Ausgaben mit Highlighting
\usepackage{caption}
\usepackage[usenames]{color}
\usepackage[english]{babel}

\definecolor{lightgray}{RGB}{230,230,230}
\definecolor{darkgrey}{RGB}{88,88,88}
\definecolor{purple}{RGB}{153,17,153}
\definecolor{darkgreen}{RGB}{3,120,65}
\definecolor{orange}{rgb}{1,0.5,0}
\lstdefinelanguage{JavaScript}{
  keywords={typeof, new, true, false, catch, function, return, null, catch, switch, var, if, in, while, do, else, case, break},
  keywordstyle=\color{blue}\bfseries,
  ndkeywords={class, export, boolean, throw, implements, import, this},
  ndkeywordstyle=\color{darkgrey}\bfseries,
  identifierstyle=\color{black},
  sensitive=false,
  comment=[l]{//},
  morecomment=[s]{/*}{*/},
  commentstyle=\color{darkgreen}\ttfamily,
  stringstyle=\color{red}\ttfamily,
  morestring=[b]',
  morestring=[b]",
  emph=[1]{prototype,methodName,createElement,append,require,writeFile,log,asynchronousFunction,series,then,post,stringify,send,listen,capturePage,getSelected,sendMessage,setTimeout,captureVisibleTab,onload,drawImage,addListener,cb,takeScreenshot,extend},
  emphstyle=[1]\color{purple},
}
\lstset{
   language=JavaScript,
   backgroundcolor=\color{lightgray},
   extendedchars=true,
   basicstyle=\footnotesize\ttfamily,
   showstringspaces=false,
   showspaces=false,
   numbers=left,
   numberstyle=\footnotesize,
   numbersep=9pt,
   tabsize=2,
   breaklines=true,
   showtabs=false,
   captionpos=b,
   frame=single
}
\selectlanguage{english}

\def\cchapter#1{{\let\cleardoublepage\relax\chapter{#1}}}

\usepackage{url}           						% URL's (z.B. in Literatur) schöner formatieren
\usepackage{hyperref} 						% sorgt für Hyperlinks in PDF-Dokumenten
\usepackage[toc]{glossaries}						% Glossary Paket einbinden
\makeglossaries
\graphicspath{./images/}	 					% Pfad zu den Bildern des Dokumentes
\linespread{1.3}

\begin{document}

% ---------------------------------------------------------------

\frontmatter
    %
% Hauptdokument
% @author Christian Bromann <contact@christian-bromann.com>
%

\thispagestyle{empty}

\begin{center}

\includegraphics[width=0.5\textwidth]{./images/tu-logo.jpg}

\vspace{1.5cm}

Faculty IV\\
\href{http://www.eecs.tu-berlin.de/menue/fakultaet_iv}{Electrical Engineering and Computer Science}

\vspace{3cm}

{\Large \textbf{Master Thesis Exposé}}\\

\vspace{1cm}

{\Huge \textbf{Design and Implementation of a Development and Test Automation Platform for HbbTV}}\\

\vspace{3cm}

{\Large \textbf{Christian Bromann}}\\
\textbf{Matr. Nummer 359957}

\vspace{1.5cm}

\parbox{1cm}{
\begin{large}
\begin{tabbing}
Supervisor: \hspace{0.5cm}\=Dipl.-Ing. Louay Bassbouss\\
Date: \> \today
\end{tabbing}
\end{large}}

\end{center}


\mainmatter
    \cchapter{Introduction}

The Hybrid Broadcast Broadband TV (short HbbTV) is an industry standard for the delivery of broadcast and broadband TV
through a single user interface based on web standards like HTML and JavaScript. It is supported by (smart) TVs and
set-top boxes across different manufacturer connected to the Internet.

\cchapter{Problem Statement}

As the standard gains more pooularity more broadcast companies expand their services to support it. Therefor the
demand for smart developer tools increases as more apps are being published. Especially automated testing tools to ensure
the quality of deployed apps are getting more important since manual tests are time consuming and error prone.

\cchapter{Expected Outcome}

The goal of this thesis is to implement an automation driver that allows to run automation scripts of arbitrary language
on arbitrary TVs, that support the HbbTV standard, in parallel. It will be based on the W3C WebDriver protocol. A testbed
will provide developers an infrastructure that can be used when connected to the Fraunhofer network or even commercialised
as a product for other broadcast companies.

\cchapter{Timeplan}

The first month will include research on the HbbTV standard as well as research on how to embed the automation driver
into the HbbTV app to communicate with the server that receives all the WebDriver commands. The next two month are
required to implement and extend the WebDriver protocol to support the automation of arbitrary TV devices. The
implementation of a testbed will take up the fourth month. Last but no least the last two month are reserved to write
the actual thesis as well as finalising the work.

\cchapter{References}

HbbTV - \url{https://www.hbbtv.org} \\
W3C WebDriver protocol (Working Draft) - \url{https://www.w3.org/TR/webdriver/}


\end{document}
