% Da die meisten Leuten an der TU deutsch als Muttersprache haben, empfiehlt es sich, das
% Abstract zusätzlich auch in deutsch zu schreiben. Man kann es auch nur auf deutsch
% schreiben und anschließend einem Englisch-Muttersprachler zur Übersetzung geben.

\thispagestyle{empty}
\vspace*{0.2cm}

\begin{center}
    \textbf{Zusammenfassung}
\end{center}

\vspace*{0.2cm}

\noindent

Hybrid Broadcast Broadband TV ist einer der letzten größeren Entwicklungen auf dem Fernsehmarkt.
Es ist ein Versuch benutzerfreundliche und erweiternde Sendeangebot an den Zuschauer zu vermitteln
über die mit dem Internet verbundene Fernsehgeräte oder Set-Top Boxen. Während sich der Standard
weiterentwickelt, wird er bereits in immer mehr Ländern in der Welt bereitgestellt. So erhalten
mehr Zuschauer Zugang zu den dargebotenen Services, die für Fernsehsender neue Vertriebsmöglichkeiten,
wie Werbung oder Bezahlfernsehen, bieten. Wegen der großen Menge an TV-Herstellern und Modellen ist
die Unterstützung der Geräte für HbbTV jedoch sehr unterschiedlich, was die Entwicklung von HbbTV
Anwendungen deutlich erschwert. Um sicherzustellen, dass diese dennoch auf den meisten Geräten
funktionieren, ist ein mühsehliger und langer manueller Test notwendig. Dies ist mit heutigen
Entwicklungsstandards allerdings nicht mehr vereinnehmbar. Die Software Industrie hat sich über die
letzten 5 Jahre von einem wasserfall-getriebenen Modell zu einem mehr agilen Anzatz bewegt, bei der
kontinuierliche Verbesserungen in kleinen Schritten eines der obersten Prinzipien ist.

Das Hauptziel dieser Master Arbeit ist es, den Entwicklungs- und Testprozess von HbbTV Applikationen
zu verbessern, indem eine Entwicklungs- und Testautomatisie-rungsplatform geschaffen wird, die
HbbTV Entwicklern hilft, die selben Probleme, die es bereits für Web und Mobile Anwendungen
seit Jahren gibt, zu lösen. Sie untersucht die aktuellen Standards in der Entwicklung und
Qualitätssicherung von Software und zeigt auf wie diese Methoden ebenfalls im Bereich des
Fernsehens angewendet werden können. In dem ein Kommunikationskanal zwischen Applikation und
modernen Entwicklungsanwendungen, wie z.B. den Chrome DevTools, geschaffen wird, ermöglicht
man Entwicklern zum ersten Mal detailierte Informationen über die Anwendung während der
Entwicklung in Erfahrung zu bringen. Zudem zeigt die Arbeit auf, wie durch die Bereitstellung
eines Automatisierungsprogammes, welches diesen Kommunikationskanal nutzt, funktionale Tests
auf echten Fernsehgeräten in automtisierter Form ausgeführt werden können.

Durch diese Technologien wird demonstriert, wie effizient die Entwicklung von HbbTV Anwendungen
sein kann. Der Entwickler erhählt, neben dem Applikationsaufbau, einer JavaScript Konsole und
einer detailierten Auflistung aller Netzwerk Pakete, nicht nur umfangreiche Information über
die Applikation an sich, sondern auch über ihren Zustand zu jedem beliebigen Zeitpunkt. Zudem
beweist das Automatisierungsprogramm wie schnell, zuverlässig und vielseitig funktionale Tests
in einem automatisierten Prozess, z.B. durch die Nutzung von Programmen wie Jenkins, ausgeführt
werden können. Im Vergleich am Ende wird bewiesen, dass dieser Ansatz für die Entwicklung von
HbbTV Anwendungen weitaus besser skalierbar, funktionaler und flexibler ist als jegliche bereits
existierende Lösung auf dem Markt.
