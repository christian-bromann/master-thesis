\chapter{Implementation\label{cha:chapter5}}

Chapter 5 describes the implementation part of your work. Don't explain every code detail but emphasize
important aspects of your implementation. This chapter will have a volume of 15-20 percent of your thesis.

This chapter describes the implementation of component X. Three systems were chosen as
reference implementations: a desktop version for Windows and Linux PCs, a Windows Mobile
version for Pocket PCs and a mobile version based on Android.

\section{Devtools Backend\label{sec:implDevtoolsBackend}}

- Frontend vs Backend scripts
- message passing

\subsection{Launcher\label{sec:launcher}}

- What purpose has the launcher
- diagram how it registered itself to the page

\subsection{Proxy\label{sec:proxy}}

- What usage has the proxy
- explain implementation details and problems during development

\section{HbbTV-Appium-Driver\label{sec:driver}}

- how does an Appium driver consist of
- How does it fit into the Ecosystem later on

\subsection{Implementation\label{sec:implDriver}}

- How is it implemented

\subsection{Grid Setup and Scaling\label{sec:setupscaling}}

- How to connect driver to Grid

\section{Setup Evaluation\label{sec:setupevaluation}}

- Raspberry vs injected
- advantages vs disadvantages
- when to use what

\section{TestBed Setup\label{sec:testbed}}

- How we integrated it in Fame TestBed
- Show off grid
- Demonstrate use cases
- How to run a Selenium test

\subsection{Deployment\label{sec:deployment}}

- How to add another TV
- How are they connected

\subsection{Continues Delivery of HbbTV Apps\label{sec:cicdhbbtvapps}}

- How to setup an HbbTV project with CI/CD running tests on TestBed

\subsection{Reporting of Test results\label{sec:reporting}}

- How to leverage universal usage thanks to common protocols
- demonstrate Allure reporting
