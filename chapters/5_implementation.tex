% Chapter 5 describes the implementation part of your work. Don't explain every code detail but emphasize
% important aspects of your implementation. This chapter will have a volume of 15-20 percent of your thesis.
%
% This chapter describes the implementation of component X. Three systems were chosen as
% reference implementations: a desktop version for Windows and Linux PCs, a Windows Mobile
% version for Pocket PCs and a mobile version based on Android.

\chapter{Implementation\label{cha:implementation}}

Since we are working in a web environment it only made sense to build all software components with a
popular web technology too: JavaScript. Both the DevTools Backend and Appium HbbTV Driver component
are written in NodeJS using a Babel\footnote{\url{https://babeljs.io/}} compiler to use the latest
EcmaScript features today. They both use Eslint\footnote{\url{http://eslint.org/}} for static code
analysis and have a Dockerfile so they can get seamlessly deployed on any system that has Docker\footnote{\url{https://www.docker.com/}}
installed.

\section{Devtools Backend\label{sec:implDevtoolsBackend}}

The DevTools Backend has a simple NodeJS project structur with a \textit{package.json} that contains
a handful of CLI commands to operate and build the project. It is stored on the Fraunhofer GitLab
server and can be cloned with git if permissions are granted. Once it was cloned you can simply install
the dependencies and build it using the NPM commands. The project requires to run a NodeJS version
of v7.4.0 as defined in \textit{.nvmrc}.

\vspace{1cm}
\begin{listing}[H]
\begin{minted}[mathescape, linenos, numbersep=8pt, gobble=0, framesep=2mm]{shell}
# clone project
$ git clone git@gitlab.fokus.fraunhofer.de:christian.bromann/devtools-backend
# install dependencies
$ cd devtools-backend
$ npm install
# build project
$ npm run build
# run server
$ npm run start
\end{minted}
\caption{Setup DevTools Backend component locally}
\label{lst:setupdevtools}
\end{listing}
\vspace{0.5cm}

With Docker this procedure is simplified down to these commands:

\vspace{1cm}
\begin{listing}[H]
\begin{minted}[mathescape, linenos, numbersep=8pt, gobble=0, framesep=2mm]{shell}
# clone project
$ git clone git@gitlab.fokus.fraunhofer.de:christian.bromann/devtools-backend
# install dependencies
$ cd devtools-backend
$ npm run docker:build
# run container
$ docker run -p 9222:9222 devtools-backend
\end{minted}
\caption{Setup DevTools Backend component with Docker}
\label{lst:setupdevtools}
\end{listing}
\vspace{0.5cm}

The build process not only compiles all server side files down to ES5 using Babel, it also compiles
all frontend related files, the instrumentation code, to one JavaScript file which gets injected
into the HbbTV app. This has the advantage that the structure and the way these files are written
don't differ from each other. Thanks to WebPack\footnote{\url{https://webpack.js.org/}} all frontend
files are bundled together so that they can get executed in the browser. This makes it super easy
to structure and share code on server and frontend side. The main server is build on top of Express\footnote{\url{https://expressjs.com/}},
a minimal and flexible web framework to build web applications in NodeJS. It allows to define API
endpoints provides useful utility features to work with Cookies and caching. All application code
can be found within the \textit{lib} directory. Next to the main server and proxy logic most of the
code is split up in backend and frontend. The backend part is responsible to route all WebSocket
communication between the remote debugging client (e.g. the DevTools application) and the frontend
automation code that gets executed on the Smart TV. It also tracks all network traffic that comes
through the proxy and returns the data to the client if desired. All backend related information
like registered pages, resource content or metadata are being held there to save the request to the
frontend.

As \textit{''page''} we define a debugging target that can be either an HbbTV app on a SmartTV or
any other web page where the instrumentation code is injected and connected to the backend. The
server can be used to debug multiple pages at the same time. Each page has an unique identifier
that allows to connect a client on the desired page. The communication between the debugging target
and the server gets initiated via Socket.io\footnote{\url{https://socket.io/}}. Socket.io is a
sophisticated NodeJS module that enables real-time bidirectional event-based communication between
server and client. It is not only super fast and efficient it also works relieable on many web
platforms. Especially on Smart TVs where not all devices have support for WebSockets Socket.io
ensures that a connection can be established by using fallback strategies like XHR polling. From
the server to the debugging client a standard WebSocket server architecture was chosen since
there are no such fallback requirements to be fulfilled. In addition to that a Socket.io server
can only connect to clients that are using Socket.io as well which is undesired because it should
be accessible by arbitrary clients. Everytime the proxy injects an instrumentation script or the
launcher registers itself to the backend a new page instance is created and two new Socket channels
are opened.

\begin{listing}[H]
\begin{minted}[mathescape, linenos, numbersep=8pt, gobble=0, framesep=2mm]{js}
export default class Page extends EventEmitter {
    constructor (io, uuid, hostname, url, title, description, metadata) {
        super()
        // ...

        this.io = io.of(`/page/${this.uuid}`)
        this.io.on('connection', ::this.connect)

        this.wss = new WebSocket.Server({
            perMessageDeflate: false,
            noServer: true
        })
    }
    // ...
}
\end{minted}
\caption{Socket Channels initiated in Page class}
\label{lst:socket}
\end{listing}

It is worth mentioning that the backend doesn't create a WebSocket server every time a page instance
gets initiated. It instead attaches itself to the Express server and creates a fake server instance
(see \textit{''noServer: true''} in \ref{lst:socket}). When a client asks for a WebSocket connection
an \textit{''upgrade''} event is emitted when the opening handshake takes place. It \textit{''upgrades
the connection from HTTP to the WebSocket protocol''}\cite{socket}.

\begin{listing}[H]
\begin{minted}[mathescape, linenos, numbersep=8pt, gobble=0, framesep=2mm]{js}
// lib/index.js
export default class DevtoolsBackend {
    constructor (host = DEFAULT_HOST, port = DEFAULT_PORT) {
        this.app = express()
        this.server = this.app.listen(port)
        this.server.on('upgrade', ::this.backend.upgradeWssSocket)
        // ...
    }
    //...
}
\end{minted}
\caption{Server Initiation with ExpressJS}
\label{lst:socket}
\end{listing}

The backend module keeps a list of all registered pages and can upgrade the connection properly if
the event was fired by the server. Depending on the unique page id (uuid) the backend connects the
debugging client to the desired page and enables the communication between client and instrumentation
code.

\begin{listing}[H]
\begin{minted}[mathescape, linenos, numbersep=8pt, gobble=0, framesep=2mm]{js}
// lib/backend/index.js
upgradeWssSocket (req, socket, head) {
    const pathname = url.parse(req.url).pathname
    for (const page of this.pages) {
        if (pathname === `/devtools/page/${page.uuid}`) {
            return page.wss.handleUpgrade(
              req, socket, head, ::page.connectWebSocket
            )
        }
    }
    socket.destroy()
}
\end{minted}
\caption{Multiple Socket Channels Registered on one Server}
\label{lst:socket}
\end{listing}

Once the connection was established the debugging client can send commands and receive events from
the instrumentation code. In most cases the backend only propagates these messages from Socket.io
to the WebSocket connection and vice versa.

\subsection{Instrumentation Script}

The frontend part of the DevTools Backend component is the instrumentation script. It gets executed
in the target environment, e.g. the HbbTV browser on the Smart TV. It connects to the backend via
Socket.io and is used to propagate page events to the debugging client as well as execute commands
to transform elments or get information about the page state. The instrumentation script should be
placed at the top of the \textit{$<head>$} block to ensure that it gets executed before everything
else. This is important because it has to capture as much events as possible. In case the HbbTV app
fails to load due to a JavaScript error it only can propagate that error to the debugging client if
it has registered an error handler before the HbbTV code is running. It also overwrites certain API
like the console object to intercept messages and events. All methods for a certain Chrome DevTools
Protocol domain are stored in one file. Listing \ref{lst:navigate} shows an example of an implemented
method\footnote{\url{https://chromedevtools.github.io/devtools-protocol/tot/Page/\#method-navigate}}
for the page domain. As mentioned above it is written using the latest EcmaScript language features
like the destructuring assignment syntax\footnote{\url{https://developer.mozilla.org/en/docs/Web/JavaScript/Reference/Operators/Destructuring_assignment}}
or the export statement\footnote{\url{https://developer.mozilla.org/en/docs/web/javascript/reference/statements/export}}.
Even though these features aren't supported in current browsers (and especially not in proprietary
HbbTV browsers of old TV models) they can still be used since the code gets compiled down to a legacy
JavaScript version. This ensures that the code is fully compatible on the target device. However this
doesn't replace missing JavaScript APIs. Therefor any APIs that have not been defined within the
EcmaScript 3 specification, which is the supported version for HbbTV 1.0, is either avoided or used
in a way that it fails silently if not available.

\begin{listing}[H]
\begin{minted}[mathescape, linenos, numbersep=8pt, gobble=0, framesep=2mm]{js}
/**
 * Navigates current page to the given URL.
 *
 * @param  {String} url  URL to navigate the page to.
 */
export function navigate ({ url }) {
    if (typeof url !== 'string') {
        return
    }
    window.location.assign(url)
    return {}
}
\end{minted}
\caption{"navigate" Method of Page Domain}
\label{lst:navigate}
\end{listing}

All domain logic is then imported into a single module to export it as central API. This pattern is called
\textit{''Barrel''} and has its origin from the TypeScript world. \textit{''A barrel is a way to rollup
exports from several modules into a single convenient module. The barrel itself is a module file that
re-exports selected exports of other modules''}\cite{barrel}.

\begin{listing}[H]
\begin{minted}[mathescape, linenos, numbersep=8pt, gobble=0, framesep=2mm]{js}
import * as CSS from './css'
import * as DOM from './dom'
import * as Debugger from './debugger'
import * as Input from './input'
import * as Network from './network'
import * as Overlay from './overlay'
import * as Page from './page'
import * as Runtime from './runtime'
import * as Target from './target'
import * as Webdriver from './webdriver'
export default {
    CSS, DOM, Debugger, Input, Network, Overlay,
    Page, Runtime, Target, Webdriver
}
\end{minted}
\caption{Barrel Module Containing All Domain Logic}
\label{lst:barrel}
\end{listing}

This allows us to import one file containing an interface to access the whole domain logic. With only a
couple of lines we can then register an listener to the Socket.io channel for each domain and execute
a method based on the message parameters.

\begin{listing}[H]
\begin{minted}[mathescape, linenos, numbersep=8pt, gobble=0, framesep=2mm]{js}
import domains from './domains'
for (let [name, domain] of Object.entries(domains)) {
    this.domains[name] = domain
    this.socket.on(name, (args) => this.dispatchEvent(domain, args))
}
\end{minted}
\caption{Register Listeners to Socket Connection}
\label{lst:barrel}
\end{listing}

The \textit{dispatchEvent} method checks if the action that is defined within the arguments is implemented
and executes it if so. It then either returns the result of the function call back to the backend and
debugging client or does nothing if no result was returned. This approach allows us to enhance the domain
logic without having to register new listeners to the socket connection.

To reference DOM nodes back and forth between instrumentation script and debugging client every node on
the page is getting referenced with a node id. As soon as the instrumentation script gets initiated and
the DOM tree is rendered it runs through the whole tree and sets an internal flag (\textit{\_nodeId} on
every node. In addition to that is saves a reference to that node internally so it can find it based
on the node id at all times. If a client fetches the document using \textit{DOM.getDocument}\footnote{\url{https://chromedevtools.github.io/devtools-protocol/tot/DOM/\#method-getDocument}}
the instrumentation script returns instead of the actual node object a representation of it based on
the Node\footnote{see lib/frontend/models/Node.js} class. This representation contains the assigned node
id which allows the client to interact (e.g. change attributes or the content) with that node. In addition
to that every instance registers a mutation observer\footnote{\url{https://dom.spec.whatwg.org/\#mutation-observers}}
on itself. This allows to listen to DOM and attributes changes on DOM nodes and helps debugging clients
to update DOM tree representations in their applications (e.g. DevTools Elements panel). Other composite
objects like promises, generators or maps are handled similarly. With the PropertyObject\footnote{see lib/frontend/models/PropertyObject.js}
class the instrumentation script converts each object into a different representation of itself to allow
data exchange over the HTTP protocol. Instead of the actual object it returns an object id that is stored
in an ObjectStore\footnote{see lib/frontend/models/ObjectStore.js} on the page. Since these objects can't
be serialized they remain there and can be referenced using the object id. Similar to how DOM nodes are
stored every object in the object store is held in memory. Since every object is passed by reference this
won't have any affect on the performance of the page.

\subsection{Launcher\label{sec:launcher}}

The Launcher script is what's used to manually inject the instrumentation logic on arbitrary platforms
without setting up a proxy. It contains next to the Socket.io client library an initialization part and
the actual instrumentation script. The initialization part is important as the script needs to register
itself as a page to the DevTools Backend component. If a proxy is used this happens usually on the server
side. To inject the launcher a script tag has to be placed at the top of the \textit{$<head>$} block so
that it gets executed first. The actual code can be received from the DevTools Backend server.

\vspace{0.3cm}
\begin{listing}[H]
\begin{minted}[mathescape, linenos, numbersep=8pt, gobble=0, framesep=2mm]{html}
<html>
<head>
    <title>My Target Page</title>
    <script src="http://<devtools-backend>:9222/scripts/launcher.js"
            data-origin="debugger">
    </script>
    <!-- ... -->
\end{minted}
\caption{Inject Launcher Script}
\label{lst:launcher}
\end{listing}

Once the script is downloaded by the browser it sends an XHR request to the server with some page metadata
(e.g. uuid, url, title, user agent, etc) asking to register this page to the backend. After that it
initates the instrumentation scripts and starts the communication to the backend via Socket.io. The uuid
can be defined manually by adding a \textit{data-uuid} attribute to the script tag. If this attribute
can't be found it uses the hostname as unique id. The \textit{data-origin} attribute is used to hide this
DOM node from the debugging client. If a client requests DOM nodes from the instrumentation script these
node will be ignored so that they don't show up in any DOM representation as they actually are not part
of the web application.

\subsection{Proxy\label{sec:proxy}}

To automate the script injection the DevTools Backend component provides a proxy that modifies the content
of certain HTTP packages. It injects the Socket.io client library and the instrumentation script inline so
that it can establish a socket connection with the backend even before the document has fully loaded.
Since all network traffic will go through the proxy it is able to track all packages that have been
requested by the app. With that information we can provide a full network report for each page load
including timeline information and data on when the first package was loaded and when loading was finished.
Since you can inspect every network package it allows you to see how certain HbbTV applications load data
from the backend as well as how it tracks your behavior on the page.

Since we only want to modify packages that contain HTML code and pass on the other packages, e.g. images
or scripts, we have to make sure that we filter every request depending on its type. This can be done in
multiple way. Just by extracting the file ending you can already identify most of the packages by its
type. Additional some requests have an \textit{Accept} header that contains the expected mime type that
is about to be returned by the server. However these information do not fully ensure that the file that
is being requested should be modified or not. Given the fact that modifying a wrong file can break the
whole app we need to have a better way to detect file types. Therefor the proxy has another middleware
registered that makes an \textit{options}\footnote{\url{https://developer.mozilla.org/en-US/docs/Web/HTTP/Methods/OPTIONS}}
request to actually decide whether to pass it forward or to load its content in order to inject the
instrumentation script. The HTTP options method can be used to not only detect available HTTP methods
it also returns the content type of the requested resource. Since the response doesn't contain any body
payload it is really lightweight.

\begin{listing}[H]
\begin{minted}[mathescape, linenos, numbersep=8pt, gobble=0, framesep=2mm]{js}
class Proxy {
    // ...
    proxyFilter (req, proxyRes, next) {
        req.target = getFullUrl(req)
        request.head(req.target).then((headers) => {
            /**
             * only proxy resource if content type is an HbbTV application
             * Note: to act as normal proxy (for other apps than hbbtv) it
             * should also allow normal html content-type
             */
            const contentType = headers['content-type']
            if (contentType && (contentType.match(/hbbtv/g) ||
                contentType.match(/text\/html;/g))
            ) {
                return next()
            }
            // ...
            // if not HbbTV application pipe request directly into response
            const requestCall = request(opts)
            return requestCall.pipe(proxyRes)
        }, (err) => /* ... */)
    }
    // ...
}
\end{minted}
\caption{Filter Proxy Request based on Response Content Type}
\label{lst:proxyFilter}
\end{listing}

By calling the \textit{next} method we pass on the request to the next middleware which actually handles
script injection. This happens only if the content type contains HbbTV or text/html though.

\section{HbbTV-Appium-Driver\label{sec:driver}}

- how does an Appium driver consist of
- How does it fit into the Ecosystem later on

\subsection{Implementation\label{sec:implDriver}}

- How is it implemented

\subsection{Grid Setup and Scaling\label{sec:setupscaling}}

- How to connect driver to Grid

\section{Setup Evaluation\label{sec:setupevaluation}}

- Raspberry vs injected
- advantages vs disadvantages
- when to use what

\section{TestBed Setup\label{sec:testbed}}

- How we integrated it in Fame TestBed
- Show off grid
- Demonstrate use cases
- How to run a Selenium test

\subsection{Deployment\label{sec:deployment}}

- How to add another TV
- How are they connected

\subsection{Continues Delivery of HbbTV Apps\label{sec:cicdhbbtvapps}}

- How to setup an HbbTV project with CI/CD running tests on TestBed

\subsection{Reporting of Test results\label{sec:reporting}}

- How to leverage universal usage thanks to common protocols
- demonstrate Allure reporting
