\chapter{Concept\label{cha:chapter4}}

Chapter 4 is usually termed 'Concept', 'Design' or 'Model'. Here you describe your approach, give a
high-level description to the architectural structure and to the single components that your solution
consists of. Use structured images and UML diagrams for explanation. This chapter will have a volume
of 20-30 percent of your thesis.

This chapter introduces the architectural design of Component X. The component consists of
subcomponent A, B and C.

In the end of this chapter you should write a specification for your solution, including
interfaces, protocols and parameters.

\section{HbbTV Runtime Environment\label{sec:hbbtvruntimeenvironment}}

- What's the lifetime cycle for HbbTV apps

\section{Functional Principles of modern Web Browser\label{sec:howbrowserwork}}

- Give an overview on how browser work these days

\section{Chrome Remote Debugging Protocol\label{sec:crdp}}

- Overview about the Remote Debugging protocol
- where is it used
- Compare to other browsers

\section{Components\label{sec:components}}

- High level overview on components

\subsection{Devtools Backend\label{sec:devtoolsbackend}}

- What role plays the devtools backend
- Get into the usage possibilities

\subsection{Appium-HbbTV-Driver\label{sec:appiumhbbtvdriver}}

- What is Appium
- Why Appium

\subsubsection{The Appium Ecosystem\label{sec:ecosystem}}

- How is Appium structured
- What kind of drivers are there

\subsubsection{Automation Driver\label{sec:driver}}

- What is an Appium Driver
- Explain role and responsibilities

\subsubsection{Selenium Grid\label{sec:grid}}

- What is the Selenium Grid
- Explain role and responsibilities

\subsection{Raspberry Pi\label{sec:pi}}

- Why is it used
- Advantages/Disadvantages

\section{Continuos Integration and Delivery\label{sec:cicd}}

- What is CI/CD
- Role of CI/CD in software development
