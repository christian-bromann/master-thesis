% Chapter 4 is usually termed 'Concept', 'Design' or 'Model'. Here you describe your approach, give a
% high-level description to the architectural structure and to the single components that your solution
% consists of. Use structured images and UML diagrams for explanation. This chapter will have a volume
% of 20-30 percent of your thesis.
%
% This chapter introduces the architectural design of Component X. The component consists of
% subcomponent A, B and C.
%
% In the end of this chapter you should write a specification for your solution, including
% interfaces, protocols and parameters.

\chapter{Concept\label{cha:concept}}

This section will explain in more details the concepts of each individual component and how they
play together. Each component individually contributes to the overall goal of implementing a
development and test automation platform to build HbbTV applications. Starting with the base
component the Devtools Backend service which will help use to with the first part of the goal.
It enables new possibilities for devs to inspect an HbbTV app and understand and debug JavaScript
problems within their code. It helps us to instrument the application to run automated WebDriver
tests with the Appium HbbTV Driver. It acts as translator between the WebDriver protocol
and the Remote Debugging Protocol. To do that on a bigger scale we need the Raspberry Pi to deploy
that driver to any TV without any manual steps. To manage all these driver we then use a Selenium
Grid and with a proper CI/CD server we can leverage that setup to test and release our HbbTV apps
faster and with more confidence.

\section{Components\label{sec:components}}

\subsection{Devtools Backend\label{sec:devtoolsbackend}}

The Devtools Backend is the main component of the overall design. It has to do most of the work and
is the only component that directly interacts with the targeted environment: the HbbTV application.
Debugging an HbbTV application these days is almost impossible. Even though some TV manufactures
provide some interfaces and APIs to connect to the TV they are hardly documented and almost different
for each TV model. To provide a tool that covers all TVs of all manufactures requires a different
approach than hooking into a native interface. Until all manufactures recognize the demand for
developers to get a better development support this won't change. We already had a similar situation
almost 7 years ago when the smart phone market started to explode and a lot of people started writing
web apps or hybrid apps for Android and iOS. At that time both vendors had almost no support for
any debugging tools that would help developers to inspect the page. A tool called \textit{Weinre}
\footnote{WEb INspector REmote - \url{http://people.apache.org/~pmuellr/weinre/docs/latest/Home.html}}
was developed and found a lot of popularity within the developer community. It enabled to debug web
pages remotely, especially on a mobile device such as a phone, for the first time. It used a special
approach that allowed to do that not only for but for all mobile environments (Android, iOS and
Hybrid web applications run by PhoneGap/Cordova). After vendors started to support remote debugging
natively due to the success of this project it became obsolete and the inventor stopped maintaining
it.

This project is the role model for the Devtools Backend. Similar how it allowed remote debug apps
for smart phones the Devtools Backend will allow it to do so with HbbTV apps on smart TVs. The idea
is simple. A script that gets executed within a targeted environment connects to a server to
exchange information and commands as utility for a 3rd party authoring tool. The concept works
independant from the environment and device. As long as the target supports basic web technology
like HTML and JavaScript it will run everywhere. So this component can not only be used to debug HbbTV
apps on Smart TVs but also to inspect any other IoT device like a fridge or a coffee machine as long
as their interfaces are based on web technologies\footnote{Infact the screen in the Tesla Model S
is build on top of a proparitary web browser and therefor all Tesla apps are build with web
technology. Theoretically the Devtools Backend can be used to debug these apps with modern web
authoring tools already today.} \footnote{An example application can be found here: \url{http://dash.time4tesla.com/}}.

The component itself consists of two subcomponents. Similar to \textit{Weinre} it has a frontend
part that takes care on instrumenting the target environment and a backend part which is a server
that initializes and manages the data traffic between target environment and authoring tool. Both
subcomponents have logic to handle methods or trigger events according to the Remote Debugging
Protocol. As stated in section \ref{sec:remotedebuggingprotocol} the Remote Debugging Protocol is
supported by all WebKit browser and has first class support for one of the most used web authoring
tools these days: the Chrome Devtools. In addition to that it is actively maintained by a dedicated
team at Google and is well documented\footnote{\url{https://chromedevtools.github.io/devtools-protocol/}}.
In order to provide as much integration without any additional effort in only makes sense to rely on
a well maintained protocol like this.

- show diagram on communication (including proxy) and explain
    - include different types of socket frameworks and why
    - middleware
- how to reverse engineer
    - domains
    - difference between page id and frame id
    - events vs methods
- launcher
    - launcher use cases
    - register process
- document which methods are important
    - register domain
    - frame load
- proxy

\subsection{Appium-HbbTV-Driver\label{sec:appiumhbbtvdriver}}

- How is Appium structured
- base driver
- mjwp
- command lifecycle (uml diagram)
- running emulator

\subsection{Raspberry Pi\label{sec:pi}}

- Why is it used
- Advantages/Disadvantages
- how to be configured
- network graph

\subsection{Selenium Grid\label{sec:grid}}

- What is the Selenium Grid
- Explain role and responsibilities

\section{Continuos Integration and Delivery\label{sec:cicd}}

- What is CI/CD
- Role of CI/CD in software development
- jenkins setup (how to create a job)
