% Chapter 6 is usually termed 'Evaluation' or 'Validation'. How did you test it? In which environment? How
% does it scale? Measurements, tests, screenshots. This chapter will have a volume of 10-15 percent of your
% thesis.
% In this chapter the implementation of Component X is evaluated. An example instance was
% created for every service. The following chapter validates the component implemented in
% the previous chapter against the requirements.
% Put some screenshots in this section! Map the requirements with your proposed solution.
% Compare it with related work. Why is your solution better than a concurrent approach from
% another organization?
% - compare the following approaches:
% suite.st
% http://www.eurofins-digitaltesting.com/test-tools/testwizard-automation-suite/

\chapter{Evaluation\label{cha:chapter6}}

Looking back to chapter \ref{cha:chapter3} describing the requirements of the final product we will
evaluate the result and the usability of it in this chapter. The goal was to create a development
and testing environment for HbbTV applications that is comparable with the state of the art of
modern web development. Building web applications for the big screen turned out to be very cumbersome
since there are no tools that help the developer to understand what is going on on the TV. Common
workarounds are self build logging overlays on the developed application which might give information
about certain variable states but don't disclose the insides of the app at all. The DevTools Backend
component is the first tool that allows HbbTV developer to actually inspect a wide variety of aspects
of an HbbTV application.

\begin{figure}[htb]
  \centering
  \hspace*{-0.7cm}
  \includegraphics[width=16cm]{elementsPanel.png}\\
  \caption{Inspecting the DOM tree of an HbbTV Application using the DevTools application}\label{fig:elementsPanel}
\end{figure}

It not only allows to look into the DOM tree of the app but also modify elements and their CSS
properties. Developer have now the opportunity to build the app directly on the TV instead of
having to implement it on a browser first to then test it on a real Smart TV. Instead building
a custom logging mechanism it automatically captures all logs from the page as well as JavaScript
errors being thrown. In addition to that the \textit{Console} tab of the DevTools application
allows to execute any random JavaScript code within the context of the HbbTV page. With that it
can inspect variables of your application during runtime. It can be used by the developer to see
which JavaScript APIs are available in the browser environment of a specific target device.

\begin{figure}[htb]
  \centering
  \hspace*{-0.7cm}
  \includegraphics[width=16cm]{consolePanel.png}\\
  \caption{Debugging the ZDF HbbTV application with the Console tab}\label{fig:consolePanel}
\end{figure}

In addition to that since the TV is running all its network traffic through the proxy on the Raspberry Pi
it automatically collects all network data in a way that it can be displayed in the DevTools application as well.
It enables developers using this tool to not only see if all network requests have been resolved successfully
on their own app but also on any arbitrary HbbTV applications that are published. This gives developers and
researchers the chance to look into the loading behavior of an app to reveal information on when certain data
is loaded and if the app is tracking the viewer behavior in any way. Like in the browser the DevTools application
as seen in figure \ref{fig:networkPanel} it shows not only a list of all network requests that have been made by the
app but also their content.

\begin{figure}[htb]
  \centering
  \hspace*{-0.7cm}
  \includegraphics[width=16cm]{networkPanel.png}\\
  \caption{Network requests being made by the Pro7 HbbTV application}\label{fig:networkPanel}
\end{figure}

Sniffing through the requests shows that many small data packages that are covered as 1px gif images are
sending user information to services like INFOnline\footnote{See request details in Annex section under
listing \ref{ioam}} or a New Relic aggregator\footnote{See Annex section under listing \ref{newrelic}} containing
data on the viewer origin, his device, its model name and other device metrics. Another interestung artifact
of data that can be observed when switching aroung pages on the Pro7 HbbTV app. Everytime a new page is
opened a request is send to \url{hbbtv-track.redbutton.de} that tracks the movement of the viewer\footnote{Also
shown in Annex section listing \ref{sniffing}}. After the page has fully loaded you can see at the bottom of the
page that the main page took about 7 seconds to fully and it downloaded 735 KB in 30 requests. This can be used
to leverage interesting metrics of the performance of the HbbTV application.

Not only on the debugging side fulfills this tool all requirements that were stated before also its testing
features raise above everything that has been done in the HbbTV industry before. Until now \textit{''most TV
device browsers don’t support WebDriver''}\cite{sengo}\footnote{The tool that is statet as HbbTV testing solution
that supports Selenium doesn't exist anymore. There are no references on the internet other than this presentation.}
but with the Appium HbbTV Driver developer can for the first time setup a grid of Smart TVs to run WebDriver
tests on them. Since it is based on the WebDriver protocol which is the industry standard for automated desktop
and mobile testing there are already hundreds of libraries available that can be used to write these automated tests.
These libraries are available in all programming languages and flavors and already known by many developers that
have experience with automated tests. As shown in listing \ref{wdioExample} an automation script can be written
in couple lines of code. These tests are fairly simple to write even for non developers. There are solutions like
Cucumber\footnote{\url{https://cucumber.io/}} that allow to abstract away the coding part of these tests by writing
feature files with acceptence sentences which are translated in actual code. This makes writing e2e tests accessible
for everyone.

\section{Comparison To Other Testing Solutions\label{sec:businessmodel}}

As of now there aren't any compatible testing solutions for HbbTV applications. The majority of developer still have
to manually deploy their HbbTV applications on a server and setup a play out to access it on a real Smart TV device.
However there is one company called Suitest that tries to simplify HbbTV testing with a click and play approach.
It provides a service where developer can register their devices using a special hardware called Candy Box. The box
is a \textit{''control unit which Suitest uses to operate TVs, Set-Top-Boxes and other devices through their infrared
port''}\cite{candybox}. In combination with a paid subscription this box can be used to connect arbitrary devices
with the Suitest cloud. The company allows then to signup on their website to create tests and manage accounts.
As described in section \ref{sec:availabletestsolutions} the tests are being clicked together by a custom web interface
where the developer can choose from a variety of action and assertion options. These tests can contain itself other
tests which allows to split up multiple small tests to a big complex test suite.
When signup as a user at Suitest the first major difference immediatelly surfaces. The company provides their software
as a service which means that it can only be used at higher scale by paying a monthly subscription. The test solution
outlined in this thesis only requires hardware that costs once-only around 50\euro. The cheapest subscription plan
at Suitest starts with 159\euro. There is also a free plan available to test the service. However this does not allow
to run tests on a real Smart TV. In order to do that a Candy Box has to be purchased for 499\euro. Given that this
box could instrument 40 devices the plan still doesn't allow leverage this. Only with a paid subscription this box
can be used properly.
The Candy Box itself is a useful piece of hardware that allows to instrument arbitrary TVs. Since it uses its infrared
port it acts as remote control and has almost every ability to control the television like a human person. Connected
to the internet and to the Suitest cloud every developer can control the TV in anyway using that box without having
to be in the same location. The Raspberry on the contrary has certain limitations when it comes to controlling the TV.
The instrumentation script of the DevTools Backend only works within a web environment. This means that the TV has to
show an HbbTV app to allow the instrumentation script to work. This means that in order to run a test on the TV you
need to turn it on and open an HbbTV application manually. Everything outside of the HbbTV context is not accessible.
It is for instance not possible to open any internal menus or test a native application. These are indeed disadvantages
against a box that uses an infrared connection to the TV. However looking at the scope of this thesis these limitations
are acceptable. In addition to that the Raspberry Pi doesn't yet use any of the APIs that are provided by the
manufacturer. Almost every TV can be accessed via an HTTP interface to gain control over certain features. These
interfaces are used to allow mobile apps to work as e.g. a second remote control. Because of the fact that the Pi is
directly connected to the Smart TV it has automatically access to these interfaces as well\footnote{As part of the
research for this thesis a small Node.JS library (\url{https://gitlab.fokus.fraunhofer.de/christian.bromann/samsungtv})
was developed that can find and control newer Samsung Smart TV in the network by using its API interface.}. That being
said, the way how the Candy Box can access the Smart TV can also be achieved using this approach.\\
Writing tests with the Suitest UI is fairly simple. As shown in figure \ref{fig:suitest} the UI provides an editor
that allows to click together a chain of actions and assertions that can be read as a normal english sentences. In
order to choose elements from the app the developer can open up a utility menu that maps the mouse movements of the
user to the app on the target device. Once the mouse hovers over an element it gets highlighted and can be choosen.
In addition to that certain assertions like the ''has property'' check automatically fetches all available attributes
the developer might be interested in. The editor in general feel slick and covers a variety of actions and assertion
that can ensure that expected state of the HbbTV application. Additional features like tagging or adding notes to
certain steps are making tests not only easy to find but also to reason about. It doesn't require any technical
experience to write, execute and maintain them. It is directed to not only app developer but also to project manager
that want to make sure that certain requirements are met and features are implemented and stable. The test environment
developed with this thesis follows a completely different goal. It is based on already established industry standards
like WebDriver and targets developer directly. Tests are not clicked together using a GUI. They have to be written in
a programming language and are usually created and maintained by the same developers that also build the HbbTV
application. To interact with an element the selector has to be known. Since usually developer are familiar with the
elements on the page it is more difficult for other people in or outside of the team to write tests. However the
DevTools Backend can give insights about the DOM structure of the application and allows to find these selectors.
Still this approach requires some technical skills.\\
The approach in general is non oppinionated and allows to create custom devlivery pipelines that is suited for any
requirements a development team can have. While test on the Suitest platform can only be scheduled in certain time
intervals or on exact times, using the Appium HbbTV Driver it is possible to run tests at any desired point in time,
e.g. when a certain release is being made or if new code is proposed and about to be merged into the code base. This
allows to create gatekeepers that prevent bad code that causes regression to be introduced.

\section{Features and Offerings\label{sec:features}}

- Explain provided features

\section{Solution Evaluation\label{sec:usab}}

- Compare solutions
- For which group of people is which solution the better choice

\section{Comparison of writing automated tests for web/mobile vs TV\label{sec:diffInWritingTests}}

- Compare Inpute devices
- JS and environement feature set


- Therefor the tool has to be based on existing standards around automated testing and development
