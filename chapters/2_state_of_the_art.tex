% Chapter 2 is usually termed 'Related Work', 'State of the Art' or 'Fundamentals'. Here you will describe
% relevant technologies and standards related to your topic. What did other scientists propose regarding
% your topic? This chapter makes about 20-30 percent of the complete thesis.
%
% This section is intended to give an introduction about relevant terms, technologies
% and standards in the field of X. You do not have to explain common technologies such
% as HTML or XML.
%
% This section describes relevant technologies, starting with X followed by Y, concluding with Z.

\chapter{State of the Art\label{cha:state_of_the_art}}

There are literally two main standards/protocols that are used in this thesis and will be described in
depth in this chapter: the HbbTV standard and the Webdriver protocol. Both are fairly new standards that
have been defined within the last 5 years. Another important protocol for this work is the Chrome Remote
Debugging protocol which is used by the automation driver to drive the Webdriver tests get data out of
the HbbTV app. This thesis tries to connect the standards to allow interoperability between them to
increase the level of integration to a wide variety of tools that has not been possible before.

\section{HbbTV\label{sec:hbbtv}}

% It's always a good idea to explain a technology or a system with a citation of a prominent
% source, such as a widely accepted technical book or a famous person or organization.

- How does the market look like
- Where it is supported
- In which contries will it be supported
- long term plans for HbbTV

- What is required to use HbbTV

\section{HbbTV Runtime Environment\label{sec:hbbtvruntimeenvironment}}

- What's the lifetime cycle for HbbTV apps

\subsection{The Standard in Detail\label{sec:hbbtvstandard}}

- Where it is defined\\
- What does the standard contain\\
- How does it differ from other standards

\subsection{Development of HbbTV Applications\label{sec:devofhbbtv}}

- How do engineer build HbbTV apps these days
- What kind of tooling does already exist

\subsection{Available test solutions\label{sec:availabletestsolutions}}

- How do engineer ensure quality of HbbTV apps

\section{Test Automation\label{sec:testautomation}}

For internal references use the 'ref' tag of LaTeX. Technology B is similar to Technology A
as described in section \ref{sec:hbbtv}.

\subsection{How Test Automation Changed the Industry\label{sec:howitchanged}}

- Study cases of how current top companies test their apps (not necessarry HbbTV)
- demonstrate how similar apps get tested these days
- outline the gap between both worlds

\subsection{History of Automated Testing\label{sec:history}}

- How Selenium and Appium has evolved

\subsection{Webdriver Protocol\label{sec:webdriver}}

- What is it
- Where is it supported
- How does it differ from previous JSONWireProtocol

\subsection{Cloud Solutions\label{sec:cloud}}

- How to do test automation in big scale
- compare different types of cloud solutions

\subsection{Tooling\label{sec:tooling}}

- What tools to use to run test automation
- What are their scope and restrictions

\subsection{Analysis of Test Automation Demand for TVs\label{sec:testautomationontv}}

- How do devices differ
- only input remote control
- what could the future bring to us?

\section{Chrome Remote Debugging Protocol}

- What is it
- for what is it used for
- what about other browser (remotedebug.org)
- somethine from https://github.com/ChromeDevTools/awesome-chrome-devtools

\subsection{Functional Principles of modern Web Browser\label{sec:howbrowserwork}}

- How browser work

\subsection{Communication between Chrome DevTools Frontend and the Browser}

- How does the devtools frontend gets all the data from the browser
- why does it not affect the performance

\section{HbbTV and Web Platform Tests}

- How HbbTV devices get tested
- how other standards get tested
