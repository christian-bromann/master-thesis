\chapter{State of the Art\label{cha:chapter2}}

Chapter 2 is usually termed 'Related Work', 'State of the Art' or 'Fundamentals'. Here you will describe
relevant technologies and standards related to your topic. What did other scientists propose regarding
your topic? This chapter makes about 20-30 percent of the complete thesis.

This section is intended to give an introduction about relevant terms, technologies
and standards in the field of X. You do not have to explain common technologies such
as HTML or XML.

\section{Technologies \label{sec:tech}}

This section describes relevant technologies, starting with X followed by Y, concluding with Z.

\subsection{HbbTV\label{sec:hbbtv}}

It's always a good idea to explain a technology or a system with a citation of a prominent
source, such as a widely accepted technical book or a famous person or organization.

Exmple: Tim-Berners-Lee describes the ''WorldWideWeb'' as follows:\\
\textit{''The WorldWideWeb (W3) is a wide-area hypermedia information retrieval initiative
aiming to give universal access to a large universe of documents.''} \cite{timwww}\\
\\
You can also cite different claims about the same term.\\

According to Bill Gates
\textit{''Windows 7 is the best operating system that has ever been released''} \cite{billgates} (no real quote)

In opposite Steve Jobs claims Leopard to be
\textit{''the one and only operating system''} \cite{stevejobs}

\subsubsection{The Standard in Detail\label{sec:hbbtvstandard}}

- Where it is defined\\
- What does the standard contain\\
- How does it differ from other standards

\subsubsection{Distribution and Support\label{sec:distandsup}}

- How does the market look like
- Where it is supported
- In which contries will it be supported
- long term plans for HbbTV

\subsubsection{Devices and Setup\label{sec:deviceandsetup}}

- What is required to use HbbTV

\subsubsection{Development of HbbTV Applications\label{sec:devofhbbtv}}

- How do engineer build HbbTV apps these days
- What kind of tooling does already exist

\subsubsection{Available test solutions\label{sec:availabletestsolutions}}

- How do engineer ensure quality of HbbTV apps

\subsection{Test Automation\label{sec:testautomation}}

For internal references use the 'ref' tag of LaTeX. Technology B is similar to Technology A
as described in section \ref{sec:hbbtv}.

\subsubsection{How Test Automation Changed the Industry\label{sec:howitchanged}}

- Study cases of how current top companies test their apps (not necessarry HbbTV)
- demonstrate how similar apps get tested these days
- outline the gap between both worlds

\subsubsection{History of Automated Testing\label{sec:history}}

- How Selenium and Appium has evolved

\subsubsection{Webdriver Protocol\label{sec:webdriver}}

- What is it
- Where is it supported
- How does it differ from previous JSONWireProtocol

\subsubsection{Cloud Solutions\label{sec:cloud}}

- How to do test automation in big scale
- compare different types of cloud solutions

\subsubsection{Tooling\label{sec:tooling}}

- What tools to use to run test automation
- What are their scope and restrictions

\subsubsection{Analysis of Test Automation Demand for TVs\label{sec:testautomationontv}}

- How do devices differ
- only input remote control
- what could the future bring to us?
