\chapter{Introduction\label{cha:introduction}}

In the late 19th century a Germand student named Paul Gottlieb Nipkow developed an electronical device
that was able to send images over the wire with the help of a rotating metal disk. This was one of the first
mechanical prototypes that suppose to become the television. With the beginning of the 20th century two types
of TVs emerged: mechanical and electronic television. Even though the mechanical type has seen a lot of
innovation by 1934, all television systems had been converted to electronic machines. These became more and
more popular in humans households so that almost 10 years later the number of U.S. homes with television sets
could be measured in the thousands and by the late 1990s more than 98\% of U.S. homes had at least one
television. The device was one of the first electronic devices that introduced a new generation of entertainment
human equiment that everyone suppose to have.\\
Despites innovations like colored displays or flat screen devices the television quickly lost the race against
the mobile phone as well as the personal computer and later laptop. Even though it is still a very popular
medium, people nowadays use their smart phone or personal computer way more frequest as it was at the end of the
20th century. One of the major reasons was the innovation of the internet. The more people were able to connect
each other and consum media over the world wide web the more became the television the less frequest choice.
The new generation of kids grows up with web media portals like YouTube\footnote{\url{https://www.youtube.com/}}
and generates billions and billions of clicks, views and revenue for advertisment companies every day.\\
Thanks to the most recent big innovations in television technology that are setup boxes and standards like
Hybrid Broadcast Broadband (short for HbbTV) the TV industries tries to keep up with other technologies by
introducing the so called smart televisions or sometime referred to as connected TV or hybrid televisions.
It is the first generation of such devices that are connected to the world wide web and therefor can provide
digital content in a non linear way for the first time.\\
Since then the market and the amount of smart TVs has been growing fast. With that the HbbTV standared evolved
and more and more broadcaster have build an app for their broadcast streams. World wide have more than 25
countries (mainly in Europe) adopted the standard and broadcaster have developed around 300 apps that can
be viewed by more than 43 million sold devices with HbbTV support\footnote{\url{https://www.hbbtv.org/news-events/hbbtv-ibc-2016-services-and-devices/}}.
Numbers are growing. It shows the tremendous potential of the market and the beginning of a paradigm shift
from linear TV stream to non linear media content ondemand.

\section{Motivation\label{sec:motivation}}

By introducing TVs to the Internet it automatically opened the door for web technologies to become standards
on these devices. While early setup boxes like Chromecast enabled some web-browsing experience, HbbTV has
been the first real technology that brought websites to the big screen. Instead of just watching a stream with
linear content, HbbTV supported devices also provide web sources that contextually fits to the viewed content
and allow the user to navigate through an app to watch non linera content ondemand.\\
These apps are webpages with JavaScript heavy functionality that are rendered in proparitary browsers. Unlike
normal browsers though there is no navigation menu or status bar. The page is rendered with a transparent background
so that you can show the TV stream in the back while navigating through the app. Depending on the TV manufacture
the embedded browser in the TV is mostly a clone to the existing desktop browsers though with less compatibility.
Early HbbTV supported devices are compatible with desktop browsers that have been shipped more than 10 years ago.
Especially since HbbTV run a lot of JavaScript to show its content in a dynamic way this makes it hard for
developers to build their apps.\\
Compared to modern web development where almost everything is made out of web components, build together in a
modular fashion using tools like React, Polymer or Angular, HbbTV apps are still build like JavaScript single
page applications from 10 years ago. Not only because the technology within the browser doesn't allow to use
latest web technologies also the developer integration with common used tools on the developers machine is due
to device boundaries not possible. Since the internet is around way longer than the fact that it is supported on
TV devices the process of web development has become an own industry and tooling around it an own market.
Companies constantly build frameworks, tools and integrations that makes building complex web apps easier and
maintainable. Especially browser vendors like Google or Mozilla are interested in providing an excellent
developing experiences as this has become the only differentiator between browsers these days. After web
technologies has been standardised by the W3 Consortium\footnote{e.g. latest HTML specification: \url{https://www.w3.org/TR/html5/}}
the compatition between browsers is not based on the question who can interpet the HTML code better and therefor
render the page more correctly but more on which browser supports the latest recommended standards and fanciest
technologies.\\
Unfortunatelly this is not the case for all embedded browser in smart TVs. The HbbTV standard was developed as
a superset of HTML and JavaScript. The specification itself recommends support for certain APIs but doesn't
require the manufacture to embed the latest browser and all their features. This is partially due to the fact
that not all web technologies are applicable on a television. For example there is no WebRTC support because
most of the TVs don't have a camera installed. Also an HbbTV application was suppose to only add contextual
information to the provided broadcast stream. Building complex web applications was never considered to become
realitiy at the first place. However as more broadcaster discovered the opportunities that HbbTV can bring to
the audience more people were interested in finding new ways to deliver interactive content to everyone in
front of the screen.\\
This yields a problem that has been around in web development for ages but almost died due to the standardisation
of web technology. With more support for the HbbTV standard the manufacture started to improve the functionaility
of embedded browser not only by adding more computing power to smart TVs but also by providing better compatibility
to already accepted web standards. Also the HbbTV standard itself evolved and requires now certain technology to be
supported in order to allow the TV to label itself as HbbTV compliant. The problem with this is that not everyone
buys a new TV as soon as there is a better one on the market. Especially since they last longer than usual mobile
phones or computer the update cycles of televisions are long. This creates a highly fragmented user market with
tons of different devices over time that all support a different level of HTML and JavaScript. Building qualitiv
HbbTV apps that suppose to run on the majority of devices in peoples households becomes super time consuming
and expensive since you don't know if the device can execute your scripts or if you've used functionality that
is not supported. As a developer you not only have to have all the TV devices which is literally impossible but you
also need to manually test your app on each one of these. This process is cumbersome and not scaleable.

\section{Objective\label{sec:objective}}

With the HbbTV standard not being older as a couple of years the development and quality ensurance process for
building apps for the smart TV is lagging behind modern web development standards. Due to the high fragmentation
of televisions on the market it is almost impossible to ensure 100\% functionality for each individual TV.
In addition to that since the browser that renders the app is embedded on a remote device it appears to be
way more difficult to not only build HbbTV applications but also to debug them in case a certain TV doesn't
run a certain functionality. In addition to that since HbbTV is a fairly new standard on the market it has not
even closed developed a community around the technology compared to the modern web on desktop and mobile. Not
more than a handful frameworks has been developed so far that can simplify the work of an HbbTV developer. Most
of the problems still have to be solved individually which increases the probability of introducing errors and
issues that have to be fixed.\\


\section{Scope\label{sec:scope}}

Here you should describe what you will do and also what you will not do. Explain a little
more specific than in the objective section. 'I will implement X on the platforms Y and Z
based on technology A and B.'

Conclude this subsection with an image describing 'the big picture'. How does your solution
fit into a larger environment? You may also add another image with the overall structure of
your component.

\section{Outline\label{sec:outline}}

The 'structure' or 'outline' section gives a brief introduction into the main chapters of
your work. Write 2-5 lines about each chapter. Usually diploma thesis are separated into
6-8 main chapters.\\
\\
\textbf{Chapter \ref{cha:chapter2}} is usually termed 'Related Work', 'State of the Art'
or 'Fundamentals'. Here you will describe relevant technologies and standards related
to your topic. What did other scientists propose regarding your topic? This chapter makes
about 20-30 percent of the complete thesis.\\
\\
\textbf{Chapter \ref{cha:chapter3}} analyzes the requirements for your component. This
chapter will have 5-10 pages.\\
\\
\textbf{Chapter \ref{cha:chapter4}} is usually termed 'Concept', 'Design' or 'Model'.
Here you describe your approach, give a high-level description to the architectural
structure and to the single components that your solution consists of. Use structured
images and UML diagrams for explanation. This chapter will have a volume of 20-30
percent of your thesis.\\
\\
\textbf{Chapter \ref{cha:chapter5}} describes the implementation part of your work. Don't
explain every code detail but emphasize important aspects of your implementation. This
chapter will have a volume of 15-20 percent of your thesis.\\
\\
\textbf{Chapter \ref{cha:chapter6}} is usually termed 'Evaluation' or 'Validation'. How
did you test it? In which environment? How does it scale? Measurements, tests, screenshots.
This chapter will have a volume of 10-15 percent of your thesis.\\
\\
\textbf{Chapter \ref{cha:chapter7}} summarizes the thesis, describes the problems that
occurred and gives an outlook about future work. Should have about 4-6 pages.
