\chapter{Conclusion\label{cha:chapter7}}

% Chapter 7 summarizes the thesis, describes the problems that occurred and gives an outlook about future
% work. Should have about 4-6 pages.
%
% The final chapter summarizes the thesis. The first subsection outlines the main ideas
% behind Component X and recapitulates the work steps. Issues that remained unsolved are
% then described. Finally the potential of the proposed solution and future work is
% surveyed in an outlook.

More and more digital devices are getting connected to the internet and run applications that are build with web technologies. The HbbTV protocol was the foundation of bringing the web on the big screen. This fairly new standard now encounters the same issues of debuggability and testability as other device and software markets before, e.g. mobile application development. Until today there is no solution that really addresses these issues. As the HbbTV standard evolves and the device fragmentation increases the demand for sophisticated tools for quality assurance rises too. This thesis solves these issues by connecting established technologies and protocols together to create a testing and debugging solution that can be integrated in modern software development pipelines.

\section{Summary\label{sec:summary}}

% Explain what you did during the last 6 month on 1 or 2 pages!

The work on this thesis started with an in-depth research phase on the HbbTV standard and current testing technologies. Unfortunately there were no existing solutions found at this point. Some companies like BBC provided some resources of their own homegrown solutions\footnote{See BBC Hive Open Source Test Tools (\url{http://bbc.github.io/hive-ci/})} that are suited for their requirements. However the goal of this work was to find a way to make testing and debugging of arbitrary HbbTV applications accessible to everyone. Therefor it has to be non opinionated and should not prescribe a certain way of how to do things. To evaluate the technical opportunities, one of the first steps was to discover the accessibility of a modern Smart TV. Using tools like WireShark a variety of interfaces could be discovered that allowed mobile applications to e.g. remote control the TV and enable second screen technology. These interfaces also gave insights on the device capabilities as well as their underlying software. Given that there were multiple manufactures with different Smart TV platforms that even have different interfaces based on their models, this approach of controlling a TV device turned out to not be a very promising strategy.

Another interesting view point were virtual machines that emulated an HbbTV environment on your operating system. With the Opera TV Developer tools\footnote{\url{http://www.operasoftware.com/products/tv/tv-developer-tools}} the company that made the Opera browser provides a set of tools and software that even allows developers to remote debug HbbTV applications from the browser. With the Opera TV Emulator it was not only possible to debug the app using the Chrome DevTools\footnote{\url{https://wiki.operatv.tv/display/OTV/Debugging}} it also allowed to run automated WebDriver tests by attaching the ChromeDriver to the Chrome DevTools Session that was run by the WebKit engine inside the emulator\footnote{\url{https://wiki.operatv.tv/display/OTV/Selenium+testing}}. The idea of being able use the Chrome DevTools, which is the most powerful tool these days to inspect modern web apps, and test HbbTV applications using the WebDriver protocol, which drives millions of automated tests on mobile and desktop every day, seemed to be really compelling as a developer. However being restricted to an emulated environment is suboptimal and would at the end not solve the problem of quality assurance in a market of high device fragmentation like in the TV market.

The requirements of today's software delivery processes have been clearly influenced by the principles of agile software development. The demand to ship qualitative software faster also applies for the TV market where advertisement campaigns and broadcast content is getting rolled out constantly. To build a pipeline that provides such quick delivery cycles it requires \textit{''an automated set of tools from code to delivery''}\cite{Lehtonen2015DefiningMF}. The tools within the delivery process have to be highly interoperable to allow to build a pipeline that fits the developers needs since every software is different. Therefor the debugging and testing platform for HbbTV has to be interoperable too, which it only can be if it relies on open standards and protocols. Selenium, which is now standardized under W3C\footnote{\url{https://www.w3.org/}} as WebDriver, is such an open standard. It was developed as open source software since 2004. As industry leader for automated testing it runs millions of tests every day in desktop and mobile environments. To build a testing tool for HbbTV that is future proofed and aligns with today's software development standards it only makes sense to build on top of this technology. The WebDriver protocol itself only defines an interface of methods and properties to automate a browser. However it doesn't specify how the automation is being implemented. Therfore there are dedicated driver softwares for each individual browser. These drivers use different approaches to execute a certain WebDriver command on the application. Interestingly the driver for the Chrome browser uses a protocol that also drives one of the most powerful tools to develop and debug modern applications in Chrome. This protocol is called Chrome DevTools Protocol\footnote{\url{https://chromedevtools.github.io/devtools-protocol/}} which describes an interface to interact with the browser on multiple different domains like DOM, CSS or Network. It is build into WebKit which is a browser engine that drives a lot of other browser including Opera and Safari. The browser that render HbbTV applications on Smart TVs are also mainly driven by WebKit. Unfortunately it can't be accessed programmatically. The protocol still provides the ideal baseline for implementing a development and testing platform since it can be integrated in tools like Chrome DevTools.

That being said the major programming effort for this thesis was to re-implement all interfaces that are described in the DevTools protocol as a standalone script. The use case model was the Chrome DevTools application. The goal was to create a script that once injected into a web environment can be used to connect to a Chrome DevTools application. To provide support for the Network domain a proxy was developed that captures network packages requested by the app. Obviously it is not possible to support all methods described in the protocol using just the JavaScript API. In fact the protocol is very comprehensive, implementing all methods would be impossible time-wise. Therefor the scope was limited to support the most important tools, e.g. the Element, Console and Network tab. By inspecting the DevTools applications using the DevTools application itself and reverse engineer the communication between the application and a normal website it was possible to adopt the sequence of events and commands to a standalone script. This allowed an interesting in depth view of how modern HbbTV applications from private broadcasters like Pro7 or public broadcasters like ARD/ZDF were build. It showed that many channels are using the same application framework and only differ in design and content based on the brand.

The last part of the thesis was to build an automation driver to run WebDriver tests on Smart TVs running HbbTV applications. Appium already established a framework for building automation drivers based on the WebDriver protocol. It is used in drivers of different platforms like iOS, Android, Windows and Mac. Similar to how ChromeDriver and SafariDriver automates their browser, the HbbTV driver uses the implemented standalone script to execute WebDriver commands on the application.

\section{Dissemination\label{sec:dissemination}}

% Who uses your component or who will use it? Industry projects, EU projects,
% open source...? Is it integrated into a larger environment? Did you publish any papers?

The Fraunhofer Institute for Open Communication Systems will use this tool in near future to build and test their HbbTV applications on daily basis. Their TV laboratory will be equipped with a handful Raspberry Pis running the Appium HbbTV Driver and DevTools Backend components to debug and run tests on different TV models. This will be the foundation of a larger Selenium Grid that will allow to run tests on multiple devices at the same time. The code itself will be documented and open sourced so that it can be improved by people that are interested in this software.

In addition to that a talk with the title \textit{''Appium for Couch Potatoes: an HbbTV Driver''} was proposed and accepted at the Selenium Conference 2017 in Berlin\footnote{Talk proposal: \url{https://www.seleniumconf.de/talks\#christian-bromann}}. It will introduce the Appium HbbTV Driver component and will show off how to run automated tests on Smart TVs. Being able to test connected devices using Appium fits perfectly into the projects vision. Therefor the thesis was already mentioned in 2016 at the Selenium Conference in London in a talk by Jonathan Lipps with the title \textit{''StarDriver Enterprise Appium to the Future''}\footnote{Video available on YouTube: \url{https://youtu.be/e61OhZzbsEI?t=16m42s}}. Other abstracts and talk proposals had been sent out but haven't got accepted until due date.

\section{Problems Encountered\label{sec:problems}}

% Summarize the main problems. How did you solve them? Why didn't you solve them?

During the implementation of the DevTools Backend component most of the problems were encountered due to the limitation of the HbbTV browsers and the restriction of not being able to debug code that is running on a Smart TV. Until today the functionality of the tool is limited for older TV generations that don't support the JavaScript APIs that have been used in the instrumentation script. These issues couldn't be solved due to the lack of time. Running a Selenium Grid without manual interaction is also currently still an issue. The Appium HbbTV Driver can only interact with a device when the instrumentation script was injected into an HbbTV app. In order to run tests each device has to be initially setup which includes starting the TV and open an HbbTV app. This has to be automated. Using the ethernet connection from the Raspberry this can be achieved with standards like Wake on LAN to boot up the TV or the rest interfaces provided by the Smart TV itself to simulate remote control events. Due to time limitation this was also skipped.

In addition to that some other interesting issues came up while developing the proxy server. The first version of it intercepted all packages which was a lot of overhead since the component was only interested in HbbTV documents in order to inject the instrumentation script. All assets only had to be tracked but not modified. In order to achieve that there has to be a filter that checks incoming requests to be forwarded or intercepted. The problem with that is that the request object itself, including its headers, doesn't always reveal information about the content type of the requested source. Also the filename can't be used to certainly determine the response. The solution for this problem was to make a HEAD request in advance. The response of this request contained the response headers which can be reliable used to determine the response type. Another issue that made some HbbTV applications fail to load were hostname changes. Some apps, e.g. the HbbTV app of RTL2, change the hostname between request and response. In this example the AIT package is referencing the HbbTV app at \texttt{\url{www.rtl2.de/hbbtv}}. Due to internal redirects the response contains a different host (\url{http://hbbtv.rtl2.de}). This affects some assets that don't contain a hostname in the URL. A JavaScript file referenced with \texttt{/js/app.js} now gets loaded from \url{www.rtl2.de/js/app.js} instead of \url{http://hbbtv.rtl2.de/js/app.js}. As result the HbbTV application can't be initialized successfully because some important assets were missing. To workaround this issue the proxy has to indentify all script and link tag references and has to make sure that they contain an absolute URI based on the response host name. This works fairly well but is definitely not the ultimate solution for this problem.

\section{Outlook\label{sec:outlook}}

% Future work will enhance Component X with new services and features that can be used ...

The current developed prototype is more or less just a proof of concept of how modern debugging and testing solutions can be applied to the TV screen. There are numerous enhancements that can be made. So far debugging and testing is limted to the HbbTV space only. However there are also native applications that need do be tested. A tighter integration into the developing frameworks of each TV platform would allow to provide some sort of test automation also for native applications\footnote{For example the TV platform Tizen OS which is deployed on Samsung devices has an interface for native applications that is also based on Chromes DevTools Protocol (\url{http://developer.samsung.com/tv/develop/getting-started/using-sdk/web-inspector/}).}. These integration are mainly hidden behind the rest API interface of the Smart TV. Unfortunately they are not well documented and have to be discovered for each individual platform.

Another really interesting opportunity would be to offer this software as a service and allow anyone access to Smart TV devices. Similar to the SaaS model of other cloud testing companies like SauceLabs\footnote{\url{http://saucelabs.com/}} or Perfecto Mobile\footnote{\url{https://www.perfectomobile.com/}} this service would allow customers to signup and access real Smart TV devices based on their subscription plan from anywhere in the world. Broadcast companies would not need to setup their own TV device lab anymore but instead just use this service. Next to remote debugging and automated testing features it could offer some sort of manual testing where the customers can choose a TV in the device lab and connect to it via a dedicated A/V channel. This can be achieved by mirroring the A/V stream of the TV to a server that can transform this into a video stream. A simpler option would be to just record the screen with a normal camera. Given that the number of countries that support HbbTV is growing and more broadcast companies are going to be interested in this, a company that would offer such service is very likely to be successful. Especially since the HbbTV technology is new and many new features haven't been implemented in TVs yet the device fragmentation will be growing and increasing the demand for automated testing.
